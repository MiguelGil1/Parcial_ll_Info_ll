\documentclass{article}
\usepackage[utf8]{inputenc}
\usepackage[spanish]{babel}
\usepackage{listings}
\usepackage{graphicx}
\graphicspath{ {images/} }
\usepackage{cite}

\begin{document}

\begin{titlepage}
    \begin{center}
        \vspace*{1cm}
            
        \Huge
        \textbf{Analisis Parcial II - Informatica II.}
            
        \vspace{0.5cm}
        \LARGE
            
        \vspace{1.5cm}
            
        \textbf{Luis Miguel Gil Rodriguez.}
        \\
        \textbf{Maverick Sossa Tobon.}
        \vfill
        \vspace{0.8cm}
            
        \Large
        Despartamento de Ingeniería Electrónica y Telecomunicaciones\\
        Universidad de Antioquia\\
        Medellín\\
        Marzo de 2021
            
    \end{center}
\end{titlepage}
\tableofcontents
\newpage
\section{Sección introductoria} \label{intro}
En este documento, podremos encontrar las ideas para abordar el parcial numero dos del curso Informatica II. Encontraremos en el detalles tales como el lgoritmo de redimensionamiento de imagenes para luego se mostrados en una martriz de LEDs de un tamaño ce 16 por 16 LEDs.
\section{Circuito.} \label{contenido}
Para la implementación del circuito en la plataforma Tinkercad, primeramente se comenzaron a realizar ensayos con una tira de 16 Neopixeles, la cual se conecta de la siguiente manera:
\begin{enumerate}
  \item La primera fila se conecta al puerto 2 del Arduino
  \item Se le suministra potencia de 5 Voltios a la tira de Neopixel procedentes del Arduino
  \item Se conecta el polo a tierra procedente del Arduino a la tira de Neopixel.
\end{enumerate}
\bibliographystyle{IEEEtran}
\bibliography{references}
\end{document}

